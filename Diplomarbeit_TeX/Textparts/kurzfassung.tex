%% start of file zusammenfassung.tex

\selectlanguage{naustrian}
\renewcommand{\abstractname}{Kurzfassung}
\begin{abstract}
    Die Kurzfassung ist ein sehr wichtiger und der vermutlich am meisten gelesene Teil eines wissenschaftlichen Dokuments.
    
    Es soll auf einer halben Seite der Inhalt der Arbeit zusammengefasst werden. Dabei soll auf die Ausgangslage, das Ziel, die Umsetzung und auch das finale Ergebnis eingegangen werden.
    
    Der Leserin oder dem Leser soll klar sein, was der Inhalt der Arbeit ist, und ob sich ein Lesen der Arbeit für die jeweiligen Recherchezwecke auszahlt. Deswegen ist auch das Ergebnis wichtig.
\end{abstract}
%% end of file zusammenfassung.tex