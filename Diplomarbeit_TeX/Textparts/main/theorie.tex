\responsible{Martina Musterfrau}

Theoretische Abhandlungen und Literaturrecherche. Es ist wichtig dass die Person die das jeweilige Kapitel verfasst hat davor im \verb|\responsible{}|-Tag angeführt ist. Bitte unbedingt zitieren!

Zitiert kann entweder direkt nach dem einem Absatz werden. Dafür wird am Ende

\section{Referenzieren und Literatur mit \LaTeX}
Prinzipiell gibt es zwei verschiedene Möglichkeiten ein Literaturverzeichnis zu erstellen: Manuell oder automatisch mit Unterstützung eines Hilfsprogrammes. Im folgenden wird das manuelle Erstellen mit der thebibliography-Umgebung erklärt, wie sie in der Diplomarbeitsvorlage angewandt wird. \cite{litKomb}

In der HTL St. Pölten zitieren wir nach dem IEEE-Stil. Wie Einträge im Inhaltsverzeichnis aussehen sollen ist unter folgendem Link nachzulesen:

\url{https://thesius.de/blog/articles/zitieren-ingenieur-ieee-din-iso-690/}

\subsection{Literaturverzeichnis}
Am Ende der Datei wird ein Abschnitt \verb|thebibliography| gesetzt. \verb|thebibliography| enthält die kompletten Informationen zu den Einträgen im Literaturverzeichnis.
\begin{Verbatim}[frame=single]
\begin{thebibliography}{laengste Labelbreite}
	\bibitem[text]{bezugspunkt}
\end{thebibliography}
\end{Verbatim}
In dem Feld \verb|laengste Labelbreite| wird etwas eingetragen, das mindestens so lang ist, wie das längste Label eines Eintrages (eine Zeile darunter).

Das optionale Argument \verb|text| kann ein Label enthalten, welches sowohl im Text als auch im Literaturverzeichnis erscheint.

Der Pflichtparameter \verb|bezugspunkt| enthält eine kurze Bezeichnung des Eintrages. Anhand dieser Bezeichnung wird ein Bezug von dem Verweis im Text zum Literaturverzeichnis erstellt.
\begin{Verbatim}[frame=single]
\begin{thebibliography}{99}
  \bibitem[1]{tietze} U. Tietze und C. Schenk, \textit{Electronic 
    circuits. handbook for design and application.} Heidelberg: 
    Springer, 2015.
  \bibitem[2]{litKomb} Wikibooks. (4.1.21) \textit{LaTeX-Kompendium: 
    Schnellkurs: Erstellen eines Literaturverzeichnisses}. [Online]. 
    Available: \url{https://de.wikibooks.org/wiki/LaTeX-Kompendium:
    \_Schnellkurs:\_Erstellen\_eines\_Literaturverzeichnisses}
\end{thebibliography}
\end{Verbatim}
Im \verb|\bibitem|-Befehl darf kein Zeilenumbruch verwendet werden.

Um die Überschrift des Literaturverzeichnisses zu ändern, kann folgender Befehl verwendet werden:
\begin{Verbatim}[frame=single]
\renewcommand{\refname}{Mein Literaturverzeichnis}
\end{Verbatim}
Das Literaturverzeichnis enthält alle Einträge, egal ob sie benutzt werden, oder nicht (im Gegensatz zu BibTeX, siehe unten).
% ----------------------------------------------------------------------

\subsection{Verweis aus dem Text}
Im Text wird hinter dem Zitat der Befehl \verb|\cite| verwendet.
\begin{Verbatim}[frame=single]
.... Im Grunde besteht kein Unterschied zwischen einem normalen Ver-
starker und einem Operationsverstarker. Beide dienen dazu, Spannungen 
oder Leistungen zu verstärken. \cite[S.~44]{tietze} ...
\end{Verbatim}
Um eine Quelle in das Literaturverzeichnis aufzunehmen, ohne dass sie explizit im Text als Quelle aufgeführt wird, ist \verb|\nocite| zu verwenden.

\nocite{tietze}