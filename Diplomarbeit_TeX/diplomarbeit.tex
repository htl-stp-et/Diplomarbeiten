  %% diplomarbeit.tex
  %% Copyright 2015 Simon M. Laube
  %
  % This work may be distributed and/or modified under the
  % conditions of the LaTeX Project Public License, either version 1.3
  % of this license or (at your option) any later version.
  % The latest version of this license is in
  %   http://www.latex-project.org/lppl.txt
  % and version 1.3 or later is part of all distributions of LaTeX
  % version 2005/12/01 or later.
  %
  % This work has the LPPL maintenance status `author maintained'.
  % 
  % The Current Maintainer of this work is S. M. Laube
  %
  % This work consists of the files listed in ./Help/files.txt

%%=====================================================%%
%% Neues Diplomarbeitstemplate der ET	   			   %%
%% Abteilung ab 2013/2014				   			   %%
%% Erstellt von Simon Michael Laube		   			   %%
%% Betreut von  Prof. Mag. Dipl.-Ing. Dr. Daniel Asch  %%
%%			    Prof. Dipl.-Ing. Dr. Wilhelm Haager	   %%
%%=====================================================%%
%% Dokumentklasse KOMA-Script Report

%% benötigt, da scrpage2 seit KOMA-Script 3.3 (April 2020) als veraltet gilt
%% Philipp Eilmsteiner 17.09.2020
\RequirePackage{scrlfile}
\ReplacePackage{scrpage2}{scrlayer-scrpage}

\documentclass[paper=a4,12pt]{scrreprt}
% Encoding UTF8
\usepackage[utf8]{inputenc}
% 8 Bit Aufloesung der Buchstaben
\usepackage[T1]{fontenc}
% Seitenraender
\usepackage[scale=0.72]{geometry}
% Spracheinstellungen
\usepackage[english, naustrian]{babel} % your native language must be the last one!!
% erweiterte Farbenpalette
\usepackage[dvipsnames]{xcolor}
% Abbildungen
\usepackage{graphicx}
% Tabellen (erweitert)
\usepackage{tabularx}
% TikZ + Circuit-TikZ (fuer Schaltungen)									
\usepackage[europeanresistors,							
			europeaninductors]{circuitikz}
% Nuetzliche TikZ Libraries
\usetikzlibrary{arrows,automata,positioning}
% Mathematikpakete!
\usepackage{amsmath,amssymb}							
%\usepackage{mathtools}	
% PDF Einbindung (zB Datenblaetter)
\usepackage{pdfpages}
% Source Code Einbindung, Setup siehe:
% http://en.wikibooks.org/wiki/LaTeX/Source_Code_Listings									
\usepackage{listings,scrhack} %scrhack vermeidet Umschaltung auf KOMA Floats..			
			
\usepackage{eurosym}
\usepackage{lscape}
% Diplomarbeitsspezifisches Package etdipa
\usepackage{etdipa}

%% Abkuerzungsverzeichnis
\usepackage[]{acronym}

%% Todos
\usepackage[]{todonotes}

%% Ganttdiagramme
\usepackage{pgfgantt}

%% Subfigures
\usepackage[lofdepth]{subfig}

%% HTL-Package (für HTL-Grün)
\usepackage{htlstp}

%%==== Definitionen fuer die Diplomarbeit ============%%
\dokumenttyp{DIPLOMARBEIT}
\title{Kernfusionsrektor}
\author{Max Mustermann \and Manuela Musterfrau}
\date{\today}
\place{St. P\"olten}
\schuljahr{2020/21}
\professor{Prof. Dr. Dipl.-Ing. Daniela Feuer, BEd}
\dipacolor{htlgruen}
%%====================================================%%


% Hyperlinks im Dokument
\usepackage[colorlinks=true,
			linkcolor=black,
			citecolor=green,
			bookmarks=true,
			urlcolor=black,
			bookmarksopen=true]{hyperref}

\begin{document}

\frontmatter

%%================ Titelseite ==========================%%
\maketitle
% Verantwortliche/Verfasser
\responsible{Stefan Deimel}
%%======================================================%%


%%================ Eidesstattliche Erklaerung ==========%%
\begin{Eid}%Unterschrift der Diplomanden hinzufuegen!
\unterschrift{Stefan Deimel}
\unterschrift{Philipp Eilmsteiner}
\unterschrift{Julia Stöger}
\end{Eid}\newpage
%%======================================================%%

%%================ Diplomandenvorstellung ==============%%
%% start of file diplomanden.tex

%% Diplomandenvorstellung:
\begin{Diplomandenvorstellung}
%% Schueler1 
\diplomand{Max Mustermann}
		  {01.03.2002 in St.Pölten}
		  {Waldstraße 3}
		  {3100 St.Pölten}
		  {\schule{2017--2022}{HTBLuVA St.Pölten, Abteilung für Elektrotechnik}
		  \schule{2012--2017}{NMS St. Pölten}}
		  {privat@mail.at}
		  {Images/portraits/bild}
\blankpage
%% Schueler2
\diplomand{Martina Musterfrau}
			{01.01.2002 in St.Pölten}
			{Waldstraße 3}
			{3100 St.Pölten}
			{\schule{2017--2022}{HTBLuVA St.Pölten, Abteilung für Elektrotechnik}
			\schule{2012--2017}{NMS St. Pölten}}
			{privat@mail.at}
			{Images/portraits/bild}
\end{Diplomandenvorstellung}

%% end of file diplomanden.tex

%%======================================================%%
\responsible{Stefan Deimel}


%%================ Danksagungen ========================%%
%% start of file danksagungen.tex

%% Danksagungen:
\begin{Danksagung}
    In der Danksagung kann frei und ohne Vorgabe jeder und jedem gedankt werden, die oder der zum Erfolg der Arbeit beigetragen hat. Sie sollte eine Seite nicht überschreiten.
\end{Danksagung}
\newpage

%% end of file danksagungen.tex
%%======================================================%%

%%================ Abstract /Zusammenfass. =============%%
%% start of file abstract.tex

\selectlanguage{english}
\begin{abstract}
    English version of the Kurzfassung.
\end{abstract}
\selectlanguage{naustrian}

%% end of file abstract.tex
\label{chap:zusammenfassung}
\responsible{Max Mustermann, Martina Musterfrau}

Nomen est omen.
%\selectlanguage{english} % necessary for English speaking users
% delete this line if your native language is German 
%%======================================================%%

%%================ Inhaltsverzeichnis ==================%%
\tableofcontents
%%======================================================%%


%Ab hier Hauptteil
\mainmatter

\appendix

%%================ Abkuerzungsverzeichnis ==============%%
%% start of file abkuerzungen.tex

% Abkuerzungsverzeichnis
\addchap{
	\iflanguage{english}{Acronyms}{Abkürzungsverzeichnis}}
\begin{acronym}[ACRONYM]
\acro{acb}[ACB]{Audio-Connect-Box}
\acro{led}[LED]{light-emitting diode}
\acro{opv}[OPV]{Operationsverstärker}
\acro{rew}[REW]{Room EQ Wizard}
\acro{rfi}[RFI]{radio frequency interference}
\acro{pla}[PLA]{polylactic acid}
\end{acronym}\newpage

%% end of file abkuerzungen.tex
%%======================================================%%


%%================ Abbildungsverzeichnis ===============%%
\setcounter{lofdepth}{2}
\dipalistoffigures
%%======================================================%%

%%================ Tabellenverzeichnis  ================%%
\setcounter{lotdepth}{2}
\dipalistoftables
%%======================================================%%

%%================ Literaturverzeichnis ================%%
\newpage
%% start of file literatur.tex

%% Literaturverzeichnis:
\begin{Literatur}
% TODO nummerierung/reihenfolge literatur anpassen
\bibitem[1]{data_PCM2906}{\textbf{TEXAS INSTRUMENTS:} \href{https://www.ti.com/lit/ds/symlink/pcm2906.pdf}{\emph{Datenblatt PCM2904/PCM2906}}. 2007 \newline [online] 04.03.2022 https://www.ti.com/lit/ds/symlink/pcm2906.pdf}

\bibitem[2]{input}{\textbf{WHITLOCK, Bill:} \emph{A new balanced audio input circuit for maximum common-mode rejection in real-world environments}. Journal of the Audio Engineering Society, 1995, 43. Jg., Nr. 6, S. 454-464.}

\bibitem[3]{phantom}{\textbf{PETROV, Petre Tzv:} \href{https://www.diyrecordingequipment.com/products/hc1}{\emph{5V DC To 48V DC Converter For Phantom Power Supplies}}. 04.03.2021, Electronicsforu. [online] 25.03.2022 https://www.electronicsforu.com/electronics-projects/5v-48v-dc-converter-phantom-power-supplies}

\bibitem[4]{fachkunde}{\textbf{BUMILLER, Horst; et al:} {\emph{Fachkunde Elektrotechnik.}} Haan-Gruiten: Verlag Europa-Lehrmittel, Nourney, Vollmer GmbH \& Company KG, 2020. -ISBN 978-3-808-53791-6. S.} 

\bibitem[5]{tietze} \textbf{U. TIETZE und C. SCHENK:} \textit{Electronic circuits. handbook for design and application.} Heidelberg: Springer, 2015.

\bibitem[6]{litKomb} \textbf{Wikibooks:} (4.1.2021) \textit{LaTeX-Kompendium: Schnellkurs: Erstellen eines Literaturverzeichnisses}. [Online]. Available: \url{https://de.wikibooks.org/wiki/LaTeX-Kompendium:\_Schnellkurs:\_Erstellen\_eines\_Literaturverzeichnisses}


\end{Literatur}

%% end of file literatur.tex
 
	 
%%======================================================%%
\end{document}
